\documentclass[11pt]{article}
\usepackage[margin=1.2in]{geometry} 
\usepackage{amsmath}
\usepackage{tcolorbox}
\usepackage{amssymb}
\usepackage{amsthm}
\usepackage{lastpage}
\usepackage{fancyhdr}
\usepackage{accents}
\usepackage{parskip}
\usepackage{setspace}
\usepackage{xcolor}
\usepackage{tabularx,booktabs}
\newcolumntype{Y}{>{\centering\arraybackslash}X}

\setstretch{1.25}
\pagestyle{fancy}
\setlength{\headheight}{40pt}

\begin{document}

\lhead{Mr. \textsc{H. Stobart}} 
\rhead{\textsc{Quant Finance Series \\ Issue 1, May-22}}
\cfoot{\thepage\ of \pageref{LastPage}}

\begin{tcolorbox}
\begin{center}
    \large
    \textsc{An Introduction to Options: \\ From Plain Vanilla to Exotics}
\end{center}
\end{tcolorbox}

\begin{center}
\textbf{Note:} \textit{This work is intended for informative and educational purposes only.}
\end{center}

\section*{1. Introduction}
The concept of derivatives pricing plays a fundamental role in the world of Quantitative Finance. A great deal of time and effort is spent trying to establish the `true' price of the derivative contract in question, both for profit seeking and risk management purposes. \par

Whilst the space of derivatives is enormous––one need only consider the fact there are contracts on such things as \textit{lean hogs} to appreciate just how broad a topic this is––there is one class that undoubtedly receives the most attention: \textbf{Options}. \par

The theory of option pricing has been extensively covered, ranging from the famous \textit{Black-Scholes-Merton} equation to more recent, sophisticated techniques for non-traditional contracts. \par

The purpose of this issue is simply to discuss, at a very high level, some of the different options contracts that exist. Thereby avoiding (for now) the mathematical elements of pricing such a contract and focusing purely on how they work in a descriptive sense.

I will assume the reader understands the underlying concept of an option and is comfortable with its associated terminology.

\section*{2. Vanilla Options}
There are three types of \textit{plain vanilla} options to consider:
\begin{itemize}
    \item \textbf{European} options;
    \item \textbf{American} options; \textit{and}
    \item \textbf{Bermudan} options.
\end{itemize}
Before we continue, it is worth noting their names have no bearing on the geographical location within which they are traded. For instance, a European option can be, and certainly is, traded in the U.S.

\subsection*{2.1 European Options}
A European option is the most straightforward. It provides the owner the right, but not the obligation, to exercise their option \textit{at expiry only}.

\textbf{Example:}
Suppose Amazon shares are currently trading at \$2500, and you hold a 1-month European call option, with a strike of \$2600. Then in one month's time, when your option expires and only then, you will have the choice of whether to exercise your right to buy those Amazon shares at \$2600.

\subsection*{2.2 American Options}
An American option provides the owner the right, but not the obligation, to exercise their option \textit{at any time prior to expiry}.

\textbf{Example:} Continuing from the European example, suppose instead you hold a 1-month American call option on Amazon shares, with the same strike of \$2600. Then you have the choice of whether to exercise your right to buy at any time over the next month. You could exercise it first thing Monday morning if you felt like it!

\subsection*{2.3 Bermudan Options}
A Bermudan option falls somewhere in between its European and American counterparts, hence the name `Bermudan' as lying between Europe and the U.S. It provides the owner the right, but not the obligation, to exercise their option \textit{at certain specified points prior to expiry}.

\textbf{Example:} Using our Amazon example once more, suppose you now hold a 6-month Bermudan call option, again with the strike of \$2600 (the increase in time to expiry is simply to illustrate the point). Then you have the choice of whether or not to exercise your right to buy only on the specific dates set out in the contract, for instance this may be the last working day of each calendar month. 

\newpage

\section*{3. Exotic Options}
Exotic options owe their name to the additional features included in their contracts. Plain vanilla options are primarily determined by the timing at which they can be exercised, and therefore only the current price is of concern. Exotics, however, can depend on a great deal of other factors including their price history. As a result, some Exotic Options are often referred to as \textit{path dependent} since their worth is directly related to the price path of the underlying.

For the most part, their existence is largely based on their hedging capabilities––with potentially detrimental exotic features meaning the option will be cheaper––but, as we will see, there is also scope for favourable features as well. 

We will consider the following Exotics:

\begin{itemize}
    \item \textbf{Asian} options;
    \item \textbf{Barrier} options;
    \item \textbf{Binary} options;
    \item \textbf{Chooser} options;
    \item \textbf{Compound} options;
    \item \textbf{Exchange} options; 
    \item \textbf{Forward Start} options; \textit{and}
    \item \textbf{Lookback} options. 
\end{itemize}

\subsection*{3.1 Asian Options}
An Asian option is not too dissimilar from its plain vanilla counterparts, except that it relies on averages. What makes Asian options so broad, however, is the number of combinations that can be generated from the \textit{type} of averaging and the \textit{component} to be averaged.

The former can either be arithmetic or geometric, whilst the latter can be the strike or the price of the underlying. That is, we perform the specific averaging to the specific component on set dates over the time period. 

\textbf{Example:} Suppose you own a 3-month Asian call option on some underlying share, with a geometrically averaged strike price. Then at the end of that 3-month period, the final strike price will be calculated by geometrically averaging the underlying stock price at specified intervals within those 3 months (for instance, the closing price every Friday). You will then exercise your option if the price at expiry is greater than the computed average strike. 

\subsection*{3.2 Barrier Options}
A barrier option has a slightly more complex dependency than other Exotic options. The barrier plays an critical role in determining \textit{whether the owner has the right} to exercise at expiry. There can be one or two barriers, which can be either higher or lower than the current price of the underlying, and can be classed as either `knock-in' or `knock-out'. 

For a knock-in, the option is worthless unless the barrier is hit, at which point it becomes a traditional plain vanilla option. A knock-out is the opposite, the option becomes worthless if the barrier is hit. Given that each of these features can be included on a call or a put, this leaves plenty of combinations. Considering just the single barrier case, we have:

\textsc{Knock-In}
\begin{itemize}
    \item Up-and-In Call
    \item Up-and-In Put
    \item Down-and-In Call
    \item Down-and-In Put
\end{itemize}

\textsc{Knock-Out}
\begin{itemize}
    \item Up-and-Out Call
    \item Up-and-Out Put
    \item Down-and-Out Call
    \item Down-and-Out Put
\end{itemize}

\textbf{Example:} Suppose this time you hold two separate barrier options. The first is an Up-and-In Call Option, with an upper barrier of £110, a strike of £115, and the underlying currently at £100. The second is a Down-and-Out Put Option, with a strike of £90, and lower barrier of £60, with the same underlying currently at £100.

Initially, the up-and-in option is worthless, as the current price of the underlying is below the upper barrier and is therefore inactive. During the life of the option if, at any point, the underlying reaches the barrier of £110, then the option become active and behaves as a plain vanilla, which would give you the right to exercise at expiration of the option.

For the down-and-out option, it is active from the outset. This means it behaves like a plain vanilla put option and presuming that the underlying does not touch the lower barrier during its life, you will have the right to exercise at expiration. If, however, the price of the underlying declines so far as to touch the £60 lower barrier then your option will become inactive at that moment and is therefore worthless. 

\subsection*{3.3 Binary Options}
A binary option, sometimes referred to as a digital option, gives the owner the right to some \textit{non-variable total} amount at expiry. That is, it can essentially be thought of as an all-or-nothing option, with the owner receiving, say, cash (referred to as cash-or-nothing) or the present value of an asset (referred to as asset-or-nothing). There is no additional value as a result of the option expiring deeper in-the-money as the payoff is either received or not. 

\textbf{Example:} Suppose you own a cash-or-nothing put on an underlying share currently trading at \$100, with a strike price of \$90, and cash payoff of \$10. Then if the underlying is below the strike at expiry, you will receive the \$10 payoff. The point here is that you will receive \$10 regardless of whether that stock declines to \$89.99 or \$20.

\subsection*{3.4 Chooser Options}
As the name suggest a chooser option gives the owner the choice, \textit{at some specific date}, of whether they want the option to be a call or a put. At this point the option then reverts to being plain vanilla. 

In the simplest case the features such as strike and time to expiry are identical for both the call and put, regardless of which is chosen. However, this can be extended to more complicated scenarios in which this isn't the case. 

\textbf{Example:} Suppose you own a 3-month simple chooser option on an underlying stock with a `choice date' in one month's time. The stock is currently trading at £100, the strike price is also set at £100. Then in one month's time you will have to make a choice of whether you want the option to become a call or a put, and it will then become a plain vanilla European option for a further 3 months.

\subsection*{3.5 Compound Options}
A compound option is where things become slightly more abstract. It is essentially an option on an option, the financial equivalent of Russian nesting dolls, if you will. Given we can have either a call or a put, we now find ourselves with four permutations, calls-on-calls, calls-on-puts, puts-on-calls, and puts-on-puts. 

The compound option behaves as you would expect, only this time the underlying is another option rather than, say, a share.

\textbf{Example:} Suppose you own a call-on-a-call compound European option. Then at the expiry of the initial (or outer) call you have the right to exercise that option and receive in its place another call option on some other underlying. 

\subsection*{3.6 Exchange Options}
An exchange option is a very straightforward concept. It provides the owner the right to exercise and \textit{exchange one asset for another}. Of course, the nature of these assets is down to the mutual agreement of the two parties involved!

\textbf{Example:} Suppose you are the holder of an exchange option on two underlying oil and gas shares, BP (which you currently own) and Royal Dutch Shell. Then at expiry you can either choose to exchange your BP shares and receive Royal Dutch Shell, if say a major incident had reduced the value of BP shares, or alternatively keep your current BP shares. 

\subsection*{3.7 Forward-Start Options}
A forward-start option simply defers the start date of plain vanilla option. It gives the owner the right to exercise in accordance with its plain vanilla counterpart \textit{once the forward-start period has passed}. Usually, the strike price is set to be the price of the underlying on the date the option becomes live. 

\textbf{Example:} Suppose you own a 6-month American put option on some underlying share with a 2-month forward-start. Then the option does not come into existence until the 2-month forward-start period has passed, at which point it would revert to a traditional plain vanilla American option and could be exercised any time within the following 6-months.

\subsection*{3.8 Lookback Options}
As the name suggests a Lookback option gives the owner the right to exercise and receive a payoff that is related to the \textit{minimum or maximum price of the underlying on its monitoring dates over the time period}. 

As with Asian options, lookback options can come in different forms depending on whether the maximum or minimum is replacing the price or the strike––with a maximum or minimum making more or less sense depending on whether we are referring to a call or a put. Beyond this the option is treated like its plain vanilla counterparts.

This means we could have a lookback call, for which the strike is fixed and the price of the underlying at expiry is its maximum over the period. Similarly, we could have a lookback put, for which the strike is the maximum of the underlying over the time period and the price of the underlying is its true price at expiry. In both cases we are seeking maximise our return by using the most favourable point of the underlying's price path. 

\textbf{Example:} Suppose you own a continuously monitored 1-month lookback call, with a strike set to the minimum of some underlying stock. To take it to the extreme imagine during that month the stock stays perfectly level at £50 every day except one, when a flash crash caused the price to drop to £10. Even though the stock has stayed at the same price for almost the entire period, the minimum is £10 and you would receive a £40 payoff. 

\section*{4. Discussion}
This brief discussion only scratches the surface of the world of options. The exotics we've seen can have more subtle features and alternative forms that I have excluded for brevity. Furthermore, there are plenty of lesser known exotic options, such as Hawaiian, Cliquet, Rainbow, and many more. For the interested reader, a simple Google search will bring a deluge of information on theses weird and wonderful financial products. 

For a more detailed discussion of the options covered above, I recommend Chapter 11 of \cite{quail2009financial}, which has formed the basis of this publication. For those with less exposure to the world of derivatives in general, I would recommend Chapter 2 of \cite{wilmott2013paul}.

\bibliographystyle{IEEEtran} 
\bibliography{bibliography} 

\end{document}
